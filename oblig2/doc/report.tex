\documentclass[11pt]{article}
\usepackage[utf8]{inputenc}
\usepackage{amsmath}
\title{INF5620 - Obligatory Exercise 2}
\author{Jens Kristoffer Reitan Markussen}
\date{\today}
\begin{document}
 	\maketitle 
	
	\section{Introduction}
	
	
	\section{Equations}
		\subsection{Partial Differential Equation}
		Two-dimensional, linear wave equation, with 
		damping,

		\begin{align*}
		\frac{\partial^2u}{\partial t^2}
		+ b\frac{\partial u}{\partial t} 
		= \frac{\partial}{\partial x} 
		\left(q(x,y)\frac{\partial u}{\partial x}
		\right) + \frac{\partial}{\partial y}
		\left(q(x,y)\frac{\partial u}{\partial y}
		\right) + f(x,y,t)
		\end{align*}
		with boundary condition,
		
		\begin{align*}
		\frac{\partial u}{\partial n} = 0		
		\end{align*}				
		
		in a rectangular spatial domain 
		with these conditions
		\begin{align*}
		\Omega = [0, L_x] \times [0, L_y]
		\end{align*}
		\begin{align*}
		u(x,y,0) = I(x,y)
		\end{align*}
		\begin{align*}
		u_t(x,y,0) = V(x,y)
		\end{align*}
						
			
		\subsection{Numerics}
		PDE in compact finite difference notation
		\begin{align*}
		[D_tD_{t}u]^n_{i,j} + b[D_{2t}u]^n_{i,j} 
		= [D_{x}\bar{q}^{x} D_{x}u]^n_{i,j} 
		+ [D_{y}\bar{q}^{y} D_{y}u]^n_{i,j}
		\end{align*}
		Central difference approximations
		\begin{align*}
		\frac{\partial^2 u}{\partial t^2} 
		\approx [D_tD_{t}u]^n_{i,j}
		= \frac{u^{n+1}_{i,j} - 2u^{n}_{i,j}
		+ u^{n-1}_{i,j}}{\Delta t^2}
		\end{align*}
		\begin{align*}
		b\frac{\partial u}{\partial t}
		\approx b[D_{2t}u]^n_{i,j}
		= b\frac{u^{n+1}_{i,j} - 2u^{n-1}_{i,j}}
		{2\Delta t}
		\end{align*}
		\begin{align*}
		\frac{\partial}{\partial x} 
		\left(q(x,y)\frac{\partial u}{\partial x}
		\right)
		\approx [D_{x}\bar{q}^{x} D_{x}u]^n_{i,j} 
		 &=\frac{1}{\Delta x}
		\left[q_{i+\frac{1}{2},j}
		\frac{u^{n}_{i+1,j}-u^{n}_{i,j}}{\Delta x}
		-q_{i-\frac{1}{2},j}
		\frac{u^{n}_{i,j}-u^{n}_{i-1,j}}{\Delta x}\right] \\
		&= \frac{1}{\Delta x^2}\left(q_{i+\frac{1}{2},j}
		(u^{n}_{i+1,j}-u^{n}_{i,j})-q_{i-\frac{1}{2},j}(
		u^{n}_{i,j}-u^{n}_{i-1,j})\right)
		\end{align*}
		\begin{align*}
		\frac{\partial}{\partial y} 
		\left(q(x,y)\frac{\partial u}{\partial y}
		\right)
		\approx [D_{y}\bar{q}^{y} D_{y}u]^n_{i,j} 
		 &=\frac{1}{\Delta y}
		\left[q_{i,j+\frac{1}{2}}
		\frac{u^{n}_{i,j+1}-u^{n}_{i,j}}{\Delta y}
		-q_{i,j-\frac{1}{2}}
		\frac{u^{n}_{i,j}-u^{n}_{i,j-1}}{\Delta y}\right] \\
		&= \frac{1}{\Delta y^2}\left(q_{i,j+\frac{1}{2}}
		(u^{n}_{i,j+1}-u^{n}_{i,j})-q_{i,j-\frac{1}{2}}(
		u^{n}_{i,j}-u^{n}_{i,j-1})\right)
		\end{align*}
		
		Relation for creating initial scheme
		\begin{align*}
		\frac{\partial u}{\partial n} \approx \frac{u^{n+1}_{i,j} - 
		2u^{n-1}_{i,j}}{2\Delta t}1 = 0
		\end{align*}	
		\begin{align*}
		u^{-1}_{i,j} = u^{1}_{i,j} 
		\end{align*}				
		
		\subsubsection*{Discretization}
		Make $q$ only valid at the grid points using the arithmetic mean
		\begin{align*}
		q_{i,j+\frac{1}{2}} &= \frac{1}{2}(q_{i,j+1} + q_{i,j}) \\
		q_{i,j-\frac{1}{2}} &= \frac{1}{2}(q_{i,j} + q_{i,j-1}) \\
		q_{i+\frac{1}{2},j} &= \frac{1}{2}(q_{i+1,j} + q_{i,j}) \\
		q_{i-\frac{1}{2},j} &= \frac{1}{2}(q_{i,j} + q_{i-1,j}) 
		\end{align*}
		
		Full approximation scheme for $u^{n+1}_{i,j}$
		\begin{align*}
		u^{n+1}_{i,j} = & \left(1 + b\frac{\Delta t}{2}\right)^{-1} 
		\biggl[2u^{n}_{i,j}+u^{n-1}_{i,j}\left(b\frac{\Delta t}{2}-1\right) \\
		& +\frac{\Delta t^2}{2\Delta x^2}((q_{i+1,j} + q_{i,j})
		(u^n_{i+1,j}-u^n_{i,j})-(q_{i,j} + q_{i-1,j})
		(u^n_{i,j}-u^n_{i-1,j})) \\
		& +\frac{\Delta t^2}{2\Delta y^2}((q_{i,j+1} + q_{i,j})
		(u^n_{i,j+1}-u^n_{i,j})-(q_{i,j} + q_{i,j-1})
		(u^n_{i,j}-u^n_{i,j-1})) \biggr]
		\end{align*}
		
		Full approximation scheme at boundary ghost cells 
		$u^{n+1}_{0,j}$, $u^{n+1}_{i,0}$, $u^{n+1}_{L_x,j}$ and 
		$u^{n+1}_{i,L_y}$, and corner cells $u^{n+1}_{0,0}$, $u^{n+1}_{L_x,0}$
		,$u^{n+1}_{L_x,L_y}$ and $u^{n+1}_{0,L_y}$
		\begin{align*}
		u^{n+1}_{0,j} = & \left(1 + b\frac{\Delta t}{2}\right)^{-1} 
		\biggl[2u^{n}_{i,j}+u^{n-1}_{i,j}\left(b\frac{\Delta t}{2}-1\right) \\
		& +\frac{\Delta t^2}{2\Delta x^2}((q_{1,j} + q_{0,j})
		(u^n_{1,j}-u^n_{0,j})-(q_{0,j} + q_{1,j})
		(u^n_{0,j}-u^n_{1,j})) \\
		& +\frac{\Delta t^2}{2\Delta y^2}((q_{0,j+1} + q_{0,j})
		(u^n_{0,j+1}-u^n_{0,j})-(q_{0,j} + q_{0,j-1})
		(u^n_{0,j}-u^n_{0,j-1})) \biggr]
		\end{align*}
		Equivalent scheme can be made for the other cells.\\
		When computing $u^{n+1}$ we required knowledge of the mesh points
		$u^{n}$ and $u^{n-1}$, which means that when computing $u^{1}$ 
		we require knowledge of $u^{0}$ and $u^{-1}$. $u^{0}$ is defined 
		by the initial conditions $I(x,y)$ and $V(x,y)$, however
		we need a modified scheme for $u^{-1}$.
		\begin{align*}
		u^{-1}_{i,j} = u^{0}_{i,j}
		& +\frac{\Delta t^2}{4\Delta x^2}((q_{i+1,j} + q_{i,j})
		(u^0_{i+1,j}-u^0_{i,j})-(q_{i,j} + q_{i-1,j})
		(u^0_{i,j}-u^0_{i-1,j})) \\
		& +\frac{\Delta t^2}{4\Delta y^2}((q_{i,j+1} + q_{i,j})
		(u^0_{i,j+1}-u^0_{i,j})-(q_{i,j} + q_{i,j-1})
		(u^0_{i,j}-u^0_{i,j-1}))
		\end{align*}
		Scheme can also be used as initial condition scheme if modified
		in the same way we modified the inner scheme.
		
	\section{Implementation}
	
	\section{Conclusion}

\end{document}
