\documentclass[11pt,a4paper]{article}
\usepackage[utf8]{inputenc}
\usepackage{amsmath}
\usepackage{amsfonts}
\usepackage{amssymb}
\usepackage{graphicx}
\author{Jens Kristoffer Reitan Markussen}
\title{Final Project INF5620,\\ Nonlinear Diffusion Equation}
\begin{document}
\maketitle

\section{Introduction}
\paragraph*{}
Final project report for INF5620 on numerical investigation and analysis of a nonlinear diffusion equation.

\section{Nonlinear diffusion model}
\paragraph*{}
The project will investigate this nonlinear diffusion equation with known coefficient $\varrho$ and known function coefficient of $\alpha(u)$. Known initial condition in space $x$ at time $0$. Naumann condition on all boundaries.
\begin{align}
& \varrho u_t = \nabla \cdot (\alpha(u)\nabla u) + f(x,t),\\
& u(x,0)=I(x),\\
& \frac{\partial u}{\partial n} = 0
\end{align}

\section{Approximation}
\paragraph*{} 
While we will use the finite elements method to approximate the PDE in space, we use the backwards euler scheme to derive an implicit approximation scheme in time.
\begin{align}
 & \left[\varrho D^-_t u  = \nabla \cdot (\alpha(u)\nabla u) + f\right]^n\\
 & \varrho \frac{u^n-u^{n-1}}{\Delta t} = \nabla \cdot (\alpha(u^n)\nabla u^n)+ f^n %\\
% & u^n - \Delta t \nabla \cdot (\alpha(u^n)\nabla u^n)+ f(x,t_n) = \varrho u^{n-1}
\end{align}
\paragraph*{}
Now that we have an expression for $u$ in time, we need to derive an approximation for $u$ in space using the finite elements method. We then need to derive a variational formulation for the spatial problem,  $F(u;v)$.
\begin{align}
% a(u,v) &= \int_\Omega (\varrho u^n v + \Delta t\alpha(u^{n})\nabla u^{n} \cdot \nabla v) dx  \\
%L(v) &= \int_\Omega \varrho u^{n-1}vdx + \Delta t \int_\Omega f^n v dx + \int_{\partial\Omega} \alpha(u^{n})\frac{\partial u}{\partial n}vds
\int_\Omega \varrho\frac{u^n-u^{n-1}}{\Delta t}vdx = -\int_\Omega\alpha(u^n)\nabla u^n \cdot \nabla v dx + \int_\Omega f^n v dx + \int_{\partial\Omega} \alpha(u^{n})\frac{\partial u}{\partial n}vds
\end{align}
\subsection{Picard iterations for $\alpha(u)$}
\paragraph*{}
The problem described above is nonlinear because of the coefficient term $a(u)$. Before we can solve the PDE we need to somehow make this term linear. This can be done by replacing the unknown variable in the nonlinear term with something known.
\paragraph*{}
To reduce the amount of indices in our equations we introduce a new notation.
\begin{align*}
u^n &: u \\
n^{n,k} &: u\_ \\
u^{n-1} &: u_1
\end{align*}
Where $k$ is the picard iteration index. 




\end{document}